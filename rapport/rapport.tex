\documentclass[11pt]{report}
\usepackage[french]{babel}
\usepackage[utf8]{inputenc}
\usepackage[T1]{fontenc}
\usepackage[top = 2cm, bottom = 2cm, left = 1cm, right = 1cm]{geometry}
\usepackage{fancyvrb}
\usepackage{mathalfa}
\usepackage{amsfonts}
\usepackage[table]{xcolor}
\usepackage{amsmath}
\usepackage{graphicx}
\usepackage{listings}
\usepackage{color}
\usepackage{comment}

\fontfamily{ptm}

\begin{document}

\selectfont

\noindent
\textbf{Hachem BENYAHIA}
~\\
\textbf{Cristian GHITU}

~\\
\begin{center}
\section*{Compte-rendu IA04 - TD4 ~\\ Simulation et Environnement ~\\ Le Sudoku}

~\\
\rule{\textwidth}{1pt}
\end{center}

~\\\\
\textbf{1. Quels sont les rôles de chaque type d'agent ?}

~\\
Il y a trois types d'agents au total. L'agent d'environnement, l'agent de simulation et l'agent d'analyse. L'agent d'environnement stocke la matrice qui représente la grille [à continuer] + schéma geogebra

~\\
\textbf{2. Quelles sont les tâches de chaque type d'agents (en termes de Behaviour simples et
composites) ?}

~\\
\textbf{3. Quel sont les types des messages échangés (request, inform, subscribe, etc.), leur
utilité et leur contenu ?}

~\\
\textbf{4. En supposant que les agents d'analyse sont situés sur différentes stations, la résolution
est-elle encore possible ?}
\end{document}